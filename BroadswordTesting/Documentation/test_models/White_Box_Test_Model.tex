\subsection{White Box Test Model}

The code located inside of the navigationLocal.js file was examined in order to determine the correctness of the output that was displayed in the Black Box test.
At the very top of the script there were a set of global variables that allowed a connection to be formed to mongo DB however, there was no variable for any NSQ connection.
A global data variable was defined which contained a hardcoded JSON formatted string of routes. This variable was then converted into a string and parsed into a JSON object which was used in all of the functions.


The code covered the following functionality:
\begin{enumerate}
\item Adding a route
\item Adding user preferences
\item Updating user preferences
\item Calculating the distance between two points
\item Finding a users preferences
\item Finding a route that was added
\item Removing user preferences
\item Removing a route
\item Establishing a connection the a persistent database (Mongo DB in this case)
\end{enumerate}


Taking a second look at the text output that was received from the previous test model phase, the output correlates to each of the functions execution and outcome
(connections established, routes added and removed etc).
There seem to be no logical errors and the code preforms its required actions which correlate to the output.
